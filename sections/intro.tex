\section{基础知识}
\label{sec:intro}

\lstMakeShortInline[
  style=style@base,
  columns=fixed]|

\begin{widepar}
为了保证本指南之实践结果的可重复性,本节列举了一部分很有可能涉及的基础知识。当然,你大可以\emph{直接跳过}这一部分,等到有需要时再来查询。
\end{widepar}

\subsection{生词}

\begin{description}
  \item[发行版] 包括引擎、宏包、字体、手册等的软件集合
  \item[\TeX{}] 高德纳教授创造的排版引擎
  \item[\LaTeX{}] 一套广泛使用的 \TeX{} 格式
  \item[\hologo{XeLaTeX}] 一款支持 Unicode 的编译引擎
  \item[文档类] 可以理解为模板
  \item[宏包] 添加扩展功能的包
\end{description}

\subsection{设备}
\label{subsec:device}

成功运行 \TeX{} 排版系统的关键在于接触到一台计算机,并且尽可能满足以下条件:

\begin{enumerate}
  \item 安装有主流操作系统\sidenote{在本教程中特指以下三种:
  \begin{itemize}
    \item \faWindows{} Windows
    \item \faLinux{} Linux
    \item \faApple{} macOS
  \end{itemize}}
  \item 连接到\href{https://www.bilibili.com/video/BV1vW411Y7oh}{英特奈},以\href{https://p.nju.edu.cn}{校园网}为佳
  \item 使用看起来没有过时的浏览器
  \item 剩余足够磁盘空间,至少为 8 GB
\end{enumerate}

此外,如果你希望在 Windows 系统中进行本地安装,还需要额外满足:

\begin{enumerate}
  \item[5.] 用户文件夹名称\sidenote{请查看 \texttt{C:\textbackslash Users},即 C 盘中的“用户”文件夹}仅包含英文字母和数字
\end{enumerate}

如果你的计算机不满足至少一则上述要求,请考虑向 IT 侠互助协会求援。\sidenote{IT 侠是一个致力于无偿解决电脑问题的校内社团,参见 \url{https://itxia.club/}。}

\subsection{命令行}
\label{subsec:terminal}

\TeX{} 本身是命令行程序。尽管编辑器对此进行了包装,但编译文档实际是通过命令行指令进行的。了解命令行操作有助于排查错误,在下文的安装测试中也会用到。

命令行操作及文件名中,请尽量不要用中文,避免空格、特殊符号。

\subsubsection{打开终端}

针对不同操作系统,可以使用如下办法打开终端\sidenote{顿号前后的操作是等价的}:

\begin{description}
  \item[\faWindows{}] 右键开始菜单\sidenote{可能显示为“命令提示符”或者“Windows Powershell”}、空白处 \kbd{Shift}\sidenote{空心的箭头是 \emph{Shift} 键} + 右键、\kbd{Windows} + \kbd{R} \& |cmd|
  \item[\faLinux{}] \kbd{Ctrl} + \kbd{Alt} + \kbd{T}
  \item[\faApple{}] \kbd{⌘} + \kbd{Space} 搜索 Terminal、可在 Finder 中添加服务
\end{description}

\subsubsection{常用命令}

\begin{description}
  \item[文件管理] |cd|、|ls/dir|、|rm/del|、|clear/cls|
  \item[\LaTeX{} 相关] |tlmgr|、|xelatex|、|latexmk|、|texdoc|
  \item[命令选项] |-h|、|--help|、|/?|
\end{description}

\subsubsection{其他要素}

\begin{itemize}
  \item 复制粘贴:\kbd{Ctrl}/\kbd{⌘} + \kbd{C}/\kbd{V}、\kbd{Ctrl}/\kbd{Shift} + \kbd{Ins}
  \item 路径连接符:斜线\sidenote{UNIX 操作系统}(|/|)或反斜线\sidenote{Windows 特色}(|\|)
  \item 换行符:LF(\lstinline{\n})或 CRLF(\lstinline{\r\n})
  \item 结束进程:\kbd{Ctrl} + \kbd{C}
\end{itemize}

\lstDeleteShortInline|

\subsection{编码}

一般地,在任何场合使用(不带 BOM 的)UTF-8 编码均是最优选择。\sidenote{你可曾感受过打开文件一团乱码的恐惧?}

\subsection{是盗版吗?}

一些比较谨慎的同学,可能会问出下面的问题:
\begin{quote}
  \kaishu 学校有没有购买相关软件的正版版权?
\end{quote}

请放心,本指南涉及的服务和软件完全免费,其中相当一部分属于自由软件的范畴。同时,也欢迎你来到 \TeX{} 的开源社区!
