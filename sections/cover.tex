\maketitle

\bigskip
\medskip

自 1978 年高德纳教授发布 \TeX{} 以来,\TeX{} 排版系统已经向出版界展示了其强大的实用功能,并坐上了学术论文出版之事实标准的宝座。这当然意味着,掌握一手 \LaTeX{} 写作技术,将是科研新星发顶刊必不可少的要素!

\medskip

然而,这篇指南\emph{无意}教你 \LaTeX{} 技术。原因很简单:\emph{软件基础决定文档写作}。在浩如烟海的教程中,本文至少有两点脱颖而出:其一,字体比较大;其二,唯一宗旨是又快又好地搞定编译环境,而且仅限在南京大学校园内。

\medskip

话不多说,先问自己两个问题!如果你决定好了,就\emph{点击相应卡片}跳转到合适的教程。

\bigskip

% https://newbedev.com/making-tikz-nodes-hyperlinkable/
\tikzset{
    hyperlink node/.style={
        alias=sourcenode,
        append after command={
            let \p1 = (sourcenode.north west),
                \p2=(sourcenode.south east),
                \n1={\x2-\x1},
                \n2={\y1-\y2} in
            node [inner sep=0pt, outer sep=0pt,anchor=north west,at=(\p1)] {\hyperref[sec:#1]{\XeTeXLinkBox{\phantom{\rule{\n1}{\n2}}}}}
        }
    }
}

\tikzstyle{decision} = [diamond, aspect=1.5, fill=njuviolet!20,
    text width=4.4em, text badly centered, minimum height=3em, minimum width=5em, inner sep=0pt]
\tikzstyle{block} = [rectangle, fill=njublue!20,
    text width=6em, text centered, rounded corners, minimum height=2.5em]
\tikzstyle{line} = [draw, -latex']
\tikzstyle{cloud} = [draw, ellipse,fill=red!20, node distance=3cm,
    minimum height=2em]
\begin{center}
\begin{tikzpicture}[node distance = 2cm, auto]

    \node [decision] (local-dist) {要不要在本地安装?};
    \node [decision, below of=local-dist, node distance=3.5cm] (which-dist) {装哪个\\版本呢?};
    \node [block, below of=which-dist, node distance=3.5cm, hyperlink node=mik] (mik) {\large \hologo{MiKTeX}};
    \node [block, left of=mik, node distance=4cm, hyperlink node=online] (online) {\large 在线服务};
    \node [block, right of=mik, node distance=4cm, hyperlink node=tl] (tl) {\large \hologo{TeX}\,Live};

    \path [line] (local-dist) -- node {\small\hls 好耶} (which-dist);
    \path [line] (local-dist) -| (online);
    \path [line] (which-dist) -- (mik);
    \path [line] (which-dist) -| (tl);
    \node [left=0.7cm of local-dist, fill=white] {\small\hls 丑拒};
    \node [above=0.7cm of online, fill=white] {\small\hls 我就喜欢速战速决};
    \node [above=0.7cm of mik, fill=white] {\small\hls 轻量一点比较好};
    \node [above=0.7cm of tl, fill=white] {\small\hls 追求又大又全的体验};
\end{tikzpicture}
\end{center}

本文源码已开放在校内 Git 平台。如果你希望在其他位置使用这篇指南的内容,请参考文末的\hyperref[sec:copyright]{\emph{版权声明}}。

\clearpage
