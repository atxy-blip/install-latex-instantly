\section{反馈}

\begin{widepar}
本教程的出发点是\emph{尽可能减少选择}。如果你在阅读中感受到了任何阻碍,譬如陌生名词、模糊表述,抑或本文内容随着软件更新而过时,请立即来项目仓库\href{https://git.nju.edu.cn/atXYblip/install-latex-instantly/-/issues/new}{提出议题}。欢迎有能力的同学直接\href{https://git.nju.edu.cn/atXYblip/install-latex-instantly/-/merge_requests/new}{发起合并请求}!

\end{widepar}

\section{声明}
\label{sec:copyright}

\begin{widepar}
\begin{center}
  \zihao{1}\vskip -1cm
  \faCreativeCommons\;
  \faCreativeCommonsBy\;
  \faCreativeCommonsSa
\end{center}

\bigskip

本指南说明文字部分采用\href{https://creativecommons.org/licenses/by-sa/4.0/deed.zh}{署名-相同方式共享 4.0 国际(CC BY-SA 4.0)}许可证。你可以随意传播或修改本文内容,前提是必须给出\emph{适当的署名},以及\emph{继承本许可}。

\medskip

其中,命令行和编码两小节的内容改编自 \href{https://github.com/stone-zeng/latex-talk}{stone-zeng / latex-talk} 讲义,在此特别鸣谢。

\smallskip

\begin{center}
  \rule{10cm}{1pt}
\end{center}

\bigskip

本指南的侧栏样式来自 \href{https://github.com/fmarotta/kaobook}{fmarotta / kaobook} 项目。模板文件 \texttt{guidehandt.cls} 采用 \href{http://www.latex-project.org/lppl.txt}{LaTeX Project Public License(版本 1.3c 或更高)}进行授权。

\begin{itemize}
  \item \textbf{正文字体}:思源宋体 +  Libertinus Serif
  \item \textbf{标题字体}:更莎黑体 +  Libertinus Sans
  \item \textbf{等宽字体}:等距更纱黑体 + CMU Typewriter
  \item \textbf{首页图示}:汉仪字研欢乐宋
\end{itemize}

\end{widepar}

